\documentclass[12pt,letterpaper]{article}
%\usepackage{fullpage}
\usepackage[top=2cm, bottom=4.5cm, left=2.5cm, right=2.5cm]{geometry}
\usepackage{amsmath,amsthm,amsfonts,amssymb,amscd}
%\usepackage{lastpage}
\usepackage{enumerate}
\usepackage{fancyhdr}
%\usepackage{mathrsfs}
\usepackage{xcolor}
\usepackage{graphicx}
\usepackage{listings}
\usepackage{hyperref}
\usepackage{float}

\hypersetup{%
  colorlinks=true,
  linkcolor=blue,
  linkbordercolor={0 0 1}
}
 
\renewcommand\lstlistingname{Code}
\renewcommand\lstlistlistingname{Code}
\def\lstlistingautorefname{Alg.}

\lstdefinestyle{C}{
    language        = C,
    frame           = lines, 
    basicstyle      = \footnotesize,
    keywordstyle    = \color{blue},
    stringstyle     = \color{green},
    commentstyle    = \color{red}\ttfamily
}

\setlength{\parindent}{0.0in}
\setlength{\parskip}{0.05in}

% Edit these as appropriate

\pagestyle{fancyplain}
\headheight 35pt
\lhead{Rahul Wankhede \\ MTech (Res), CDS}                 % <-- Comment this line out for problem sets (make sure you are person #1)
\chead{\textbf{\Large Homework-6}}
\rhead{DS 288 \\ Due: Nov 21st, 2019}
\lfoot{}
\cfoot{}
\rfoot{\small\thepage}
\headsep 1.5em

\begin{document}

\section*{Problem 1}

I have found a time step size of 1/16 years to be appropriate for this problem. The plot is as follows:

\begin{figure}[H]
\centering
\includegraphics[scale=0.7]{"../hw6_1"}
\caption{RK-4 for h = 1/16 years}
\end{figure}

Increasing the time-step size by a factor of 2, 4, 8, and 16, we get the following plot for the relative errors at N(10) vs $h^{-1}$

\begin{figure}[H]
\centering
\includegraphics[width=.5\textwidth]{"../hw6_2"}\hfill
\includegraphics[width=.5\textwidth]{"../hw6_3"}\\
\caption{Loglog plot of Relative error vs $h^{-1}$}
\end{figure}

Plotting this with an equal aspect ratio, we can see that the graph is a straight line with slope (magnitude) approximately = 4, which is what we'd expect with the RK-4 method.


\newpage

\section*{Problem 2}

The maximum allowable step sizes for the different methods are given in the table ranked in decreasing order of h(most to least stable):

\begin{center}
\begin{tabular}{c c}
\hline
Method						&		h\\
\hline
Adams Moulton				&		2.0\\
Adams Predictor/Corrector	&		0.8\\
Midpoint rule				&		0.66\\
Adams Bashforth				&		0.33\\
\hline
\end{tabular}
\end{center}

For the predictor corrector method, the plots for different values of h are as follows:

\begin{figure}[H]
\centering
\includegraphics[scale=0.65]{"../h=01"}
\caption{h = 0.1 (Stable)}
\end{figure}

\begin{figure}[H]
\centering
\includegraphics[scale=0.65]{"../h=1"}
\caption{h = 1 (Unstable)}
\end{figure}

Another plot for h=0.3 is one which is stable but not very accurate.

\begin{figure}[H]
\centering
\includegraphics[scale=0.65]{"../h=03"}
\caption{h = 0.3 (Stable but not very accurate)}
\end{figure}

The stability analysis is done by hand which I am attaching here as images:

\begin{figure}[H]
\centering
\includegraphics[scale=0.55]{"../im1"}
\end{figure}

\begin{figure}[H]
\centering
\includegraphics[scale=0.55]{"../im2"}
\end{figure}


\begin{figure}[H]
\centering
\includegraphics[scale=0.55]{"../im3"}
\end{figure}


\begin{figure}[H]
\centering
\includegraphics[scale=0.55]{"../im4"}
\end{figure}



\end{document}